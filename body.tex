
%% ---------------------------------------------------------------------------
% This presentation is separated by sections and subsections

\begin{frame}[fragile]{\TeX 与 \LaTeX 的起源}
  \begin{columns}[T]
    \column{.8\textwidth}
    \begin{itemize}
      \item \TeX: $\tau\varepsilon\chi$ (\textipa{/'tEx/},
        \textipa{/'tEk/})
        \begin{itemize}
          \item 生成精美图书的排版系统
          \item 最初由 高德纳 (Donald E.~Knuth) 于 1978 年开发
          \item 最新版本为 \TeX\ 3.14159265
          \item 漂亮、美观、稳定、通用
          \item 尤其擅长数学公式排版
        \end{itemize}

        \vspace{2em}
      \item \LaTeX\ (\textipa{/'la:tEx/}, \textipa{/'leItEk/})
        \begin{itemize}
          \item Leslie Lamport 开发的一种 \TeX{} 格式
          \item 在 \TeX 的基础上提供宏包, 降低使用门槛
          \item 极其丰富的宏包,提供扩展功能
          \item 广泛用于学术界,期刊会议论文模板
        \end{itemize}
    \end{itemize}
    \column{.2\textwidth}
    % \vspace*{5mm}
    \includegraphics[width=\textwidth]{Knuth.jpg}

    % \vspace*{5mm}
    \includegraphics[width=\textwidth]{Lamport.jpg}

  \end{columns}
\end{frame}

\begin{frame}[fragile]{\LaTeX 的好处与坏处}
    \textbf{好处}
    \begin{itemize}
        \item 数学公式排版优雅 \quad $\mathcal{F}(\xi)=\int_{-\infty}^{\infty} f(x)\mathrm{e}^{-\mathrm{j}2\pi \xi x}\,\mathrm{d}x$
        \item 内容与格式分离
        \item 随心所欲的宏定义与自定义命令 \texttt{\textbackslash newcommand},\texttt{\textbackslash def}
    \end{itemize}

    \vspace{2em}
    \textbf{坏处}
    \begin{itemize}
        \item 得到易读的版本,需要编译
        \item 输入相对 Word 繁琐
        \item 非开箱即用。有时自行解决编辑器、宏包,甚至是编译错误。
    \end{itemize}

\end{frame}

\def\texlive{\TeX \ Live}
\def\mactex{Mac\TeX}

\begin{frame}[fragile]{选择发行版 -> 下载 -> 安装}
  \begin{itemize}
    \item Windows or Linux -> \texlive
      \begin{itemize}
        \item 下载 \texlive 离线安装镜像,每年 4月发布当年版本 \url{https://mirrors.sustech.edu.cn/CTAN/systems/texlive/Images/texlive.iso}
        \item 解压或挂载下载的 ISO,运行 \texttt{install-tl-windows.bat} (Windows) or \texttt{install-tl} (Linux)
        \item 切换默认仓库为国内镜像可加速今后升级
      \end{itemize}
    \item macOS -> \mactex
      \begin{itemize}
        \item $\approx$ \texlive 在 Mac 下重新封装版本
        \item 需要下载独立的安装包 \url{https://mirrors.sustech.edu.cn/CTAN/systems/mac/mactex/MacTeX.pkg}
      \end{itemize}
\end{itemize}
    \emph{不推荐安装 C\TeX 套装}
    \begin{itemize}
        \item \emph{存在严重 bug,并且完全过时(2012年已经停止维护)。}
    \end{itemize}
\end{frame}

\begin{frame}[fragile]{太麻烦!用在线的}

    \begin{itemize}
        \item 通过在线平台编辑、编译
        \item 免去安装/升级等一系列烦恼 可以多人协作 支持中文,但有时需要自己上传字体
        \item 可以多人协作
        \item 支持中文,但有时需要自己上传字体
    \end{itemize}

    \begin{itemize}
      \item Overleaf
      \begin{itemize}
          \item \url{https://www.overleaf.com}
      \end{itemize}
      \item ShareLaTex by 计算机研究协会
      \begin{itemize}
        \item \url{https://sharelatex.cra.moe/}
    \end{itemize}
      \end{itemize}
  \end{frame}


  \begin{frame}[fragile]{文件结构}
    \lstset{language=[LaTeX]TeX}
    \begin{lstlisting}[basicstyle=\ttfamily]
  \documentclass[a4paper]{article}
  % 文档类型,如 article,[]内是选项,如 a4paper
  % 这里开始是导言区
  \usepackage{graphicx} % 引用宏包
  \graphicspath{{fig/}} % 设置图片目录
  % 导言区到此为止
  \begin{document}
  这里开始是正文
  \end{document}\end{lstlisting}
  \end{frame}
  
  \begin{frame}[fragile]{\LaTeX{}“命令”}
    \framesubtitle{\emph{宏} (Macro)、或者\emph{控制序列} (control sequence)}
  \begin{itemize}
  \item 简单命令
    \begin{itemize}
      \item \verb|\命令|\hspace{2em}
      \verb|{\songti 中国人民解放军}| ~$\Rightarrow$ {\songti 中国人民解放军}
    \item \verb|\命令[可选参数]{必选参数}|\\
  \verb|\section[精简标题]{这个题目实在太长了放到目录里面不太好看}|\\
  $\Rightarrow$ {\heiti 1.1 \hspace{1em} \songti 这个题目实在太长了放到目录里面不太好看}
    \end{itemize}
  \item 环境
    \begin{columns}[c]
    \begin{column}{0.45\textwidth}
      \begin{lstlisting}[basicstyle=\ttfamily]
  \begin{equation*}
    a^2-b^2=(a+b)(a-b)
  \end{equation*}\end{lstlisting}
  \end{column}\hspace{1em}
    \begin{column}{0.45\textwidth}
  $ a^2-b^2=(a+b)(a-b)$
  \end{column}
    \end{columns}
  \end{itemize}
  \end{frame}
  
  \begin{frame}[fragile]{\LaTeX{} 常用命令}
    \begin{exampleblock}{简单命令}
  \centering
  \footnotesize
    \begin{tabular}{llll}
      \cmd{chapter} & \cmd{section} & \cmd{subsection} & \cmd{paragraph} \\
      章 & 节 & 小节 & 带题头段落 \\\hline
      \cmd{centering} & \cmd{emph} & \cmd{verb} & \cmd{url} \\
     居中对齐         &  强调      & 原样输出   & 超链接 \\\hline
    \cmd{footnote} & \cmd{item} & \cmd{caption} & \cmd{includegraphics} \\
     脚注 & 列表条目 & 标题 & 插入图片 \\\hline
    \cmd{label} & \cmd{cite} & \cmd{ref} \\
    标号 & 引用参考文献 & 引用图表公式等\\\hline
    \end{tabular}
  \end{exampleblock}
  \end{frame}

  \begin{frame}{\LaTeX{}命令举例}
    \cmdxmp{chapter}{前言}{\heiti 第 1 章\hspace{1em} 前言}
    \cmdxmp{section[精简标题]}{这个题目实在太长了放到目录里面不太好看}{\heiti 1.1
      \hspace{1em} 这个题目实在太长了放到目录里面不太好看}
    \cmdxmp{footnote}{我是可爱的脚注}{前方高能\footnote{我是可爱的脚注}}
    \end{frame}
    
  
  \begin{frame}[fragile]{\LaTeX{} 常用命令}
  \begin{exampleblock}{环境}
  \centering
  \footnotesize
  \begin{tabular}{lll}
    \env{table} & \env{figure} & \env{equation}\\
    表格 & 图片 & 公式 \\\hline
    \env{itemize} & \env{enumerate} & \env{description}\\
    无编号列表 & 编号列表 & 描述 \\\hline
  \end{tabular}
  \end{exampleblock}
  \end{frame}
  
  \begin{frame}[fragile]{\LaTeX{} 环境举例}
    \vspace{1em}
      \begin{minipage}{0.4\linewidth}
        \begin{lstlisting}[basicstyle=\ttfamily\small]
    \begin{itemize}
      \item 一条
      \item 次条
      \item 这一条可以分为 ...
        \begin{itemize}
          \item 子一条
        \end{itemize}
    \end{itemize}\end{lstlisting}
      \end{minipage}\hspace{1.5cm}
      \begin{minipage}{0.4\linewidth}
    \begin{itemize}
      \item 一条
      \item 次条
      \item 这一条可以分为 ...
        \begin{itemize}
          \item 子一条
        \end{itemize}
    \end{itemize}
      \end{minipage}
    % \smallskip
    
    \begin{minipage}{0.4\linewidth}
    \begin{lstlisting}
    \begin{enumerate}
      \item 一条
      \item 次条
      \item 再条
    \end{enumerate}\end{lstlisting}
      \end{minipage}\hspace{1.5cm}
      \begin{minipage}{0.4\linewidth}
        \vspace{-1cm}
    \begin{enumerate}
      \item 一条
      \item 次条
      \item 再条
    \end{enumerate}
      \end{minipage}
    \end{frame}
    %
    
  \begin{frame}[fragile]{列表与枚举}
    \begin{columns}
    \begin{column}{.6\textwidth}
    \begin{lstlisting}[basicstyle=\ttfamily\small]
    \begin{enumerate}
    \item \LaTeX{} 好处都有啥
      \begin{description}
        \item[好用] 体验好才是真的好
        \item[好看] 强迫症的福音
        \item[开源] 众人拾柴火焰高
      \end{description}
    \item 还有呢?
      \begin{itemize}
        \item 好处 1
        \item 好处 2
      \end{itemize}
    \end{enumerate}
    \end{lstlisting}
    \end{column}
    \begin{column}{.4\textwidth}
    {\small
    \begin{enumerate}
    \item \LaTeX{} 好处都有啥
      \begin{description}
        \item[好用] 体验好才是真的好
        \item[好看] 治疗强迫症
        \item[开源] 众人拾柴火焰高
      \end{description}
    \item 还有呢?
      \begin{itemize}
        \item 好处 1
        \item 好处 2
      \end{itemize}
    \end{enumerate}
    }
    \end{column}
    \end{columns}
    
    \end{frame}
    


        
    \begin{frame}[fragile]{\LaTeX{} 数学公式}
        \begin{itemize}
        \item 数学公式排版是 \LaTeX{} 的绝对强项
        \item 数学排版需要进入数学模式,引用 \texttt{amsmath} 宏包
            \begin{itemize}
            \item 用单个美元符号(\verb|$|) 包围起来的内容是 {\bf 行内公式}
          \item 用两个美元符号(\verb|$$|) (不推荐)或 \verb|\[ \]| 包围起来的是 {\bf 单行公式} 或 {\bf 行间公式}
            \item 使用数学环境,例如 \texttt{equation} 环境内的公式会自动加上编号,
                \texttt{align} 环境用于多行公式(例如方程组、多个并列条件等)
          \end{itemize}
        \item 寻找符号
            \begin{itemize}
              \item 运行 \texttt{texdoc symbols} 查看符号表
              \item S. Pakin. \emph{The Comprehensive \LaTeX{} Symbol List}
                    \url{https://ctan.org/pkg/comprehensive}
              \item 手写识别(有趣但不全):Detexify \url{http://detexify.kirelabs.org}
            \end{itemize}
        \item MathType 也可以使用和导出 \LaTeX{} 公式(不推荐)
        \end{itemize}
        \end{frame}
    
        
    \begin{frame}[fragile]{\LaTeX{} 数学公式}
    
    \begin{columns}
    \begin{column}{.5\textwidth}
    \begin{lstlisting}[basicstyle=\ttfamily\small]
    $V = \frac{4}{3}\pi r^3$
    
    \[
      V = \frac{4}{3}\pi r^3
    \]
    
    \begin{equation}
    \label{eq:vsphere}
    V = \frac{4}{3}\pi r^3
    \end{equation}
    \end{lstlisting}
    \end{column}
    
    \begin{column}{.5\textwidth}
    $V = \frac{4}{3}\pi r^3$
    
    \[
      V = \frac{4}{3}\pi r^3
    \]
    
    \begin{equation}
    \label{eq:vsphere}
    V = \frac{4}{3}\pi r^3
    \end{equation}
    \end{column}
    \end{columns}
    
    \end{frame}

    
    
    \begin{frame}[fragile]{层次与目录生成}
    \begin{columns}
    \begin{column}{.6\textwidth}
    
    \begin{lstlisting}[basicstyle=\ttfamily\small]
    \tableofcontents % 这里是目录
    \part{有监督学习}
    \chapter{支持向量机}
    \section{支持向量机简介}
    \subsection{支持向量机的历史}
    \subsubsection{支持向量机的诞生}
    \paragraph{一些趣闻}
    \subparagraph{第一个趣闻}
    \end{lstlisting}
    \end{column}
    \begin{column}{.4\textwidth}
    第一部分\quad 有监督学习\\
    第一章\quad 支持向量机 \\
    1. 支持向量机简介 \\
    1.1 支持向量机的历史 \\
    1.1.1 支持向量机的诞生 \\
    一些趣闻  \\
    第一个趣闻
    \end{column}
    \end{columns}
    
    \end{frame}
    
    